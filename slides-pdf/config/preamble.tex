%-----------------------------------------------------------------------

\usepackage[math, light]{iwona}
% \usepackage{cmbright}
% \usepackage[charter]{mathdesign}
% \usepackage{palatino}

% \usepackage[sfdefault, light, condensed]{roboto}
% \usepackage[default, light]{sourcesanspro}

% \usepackage[default]{lato}
% \AtBeginDocument{\fontseries{l}\selectfont} % Lato Light.

% Font for code. ----------------------------
% \usepackage[scaled=.75]{beramono}
\usepackage{inconsolata}
% \usepackage[scaled]{ulgothic}

%-----------------------------------------------------------------------

% \usepackage{lmodern}
\usepackage{amssymb, amsmath}
% \usepackage{ifxetex, ifluatex}
\usepackage{fixltx2e} % provides \textsubscript
\usepackage[utf8]{inputenc}
\usepackage[shorthands=off,main=brazil]{babel}
\usepackage{graphicx}

\usepackage{xcolor}
\usepackage{setspace}
\usepackage{xspace}
\usepackage{tabularx}


\setlength{\parindent}{0pt}
\setlength{\parskip}{6pt plus 2pt minus 1pt}
\setlength{\emergencystretch}{3em}  % prevent overfull lines
\providecommand{\tightlist}{%
  \setlength{\itemsep}{0pt}\setlength{\parskip}{0pt}}
\setcounter{secnumdepth}{0}

%-----------------------------------------------------------------------

\usepackage[hang]{caption}
\captionsetup{font=footnotesize,
  labelfont=footnotesize,
  labelsep=period}
\usepackage{subfig}

\providecommand{\tightlist}{%
  \setlength{\itemsep}{0pt}\setlength{\parskip}{0pt}}

%-----------------------------------------------------------------------

\usepackage{tikz}
\usebackgroundtemplate{
  \tikz[overlay, remember picture]
  \node[% opacity=0.3,
        at=(current page.south east),
        anchor=south east,
        inner sep=0pt] {
          \includegraphics[height=\paperheight, width=\paperwidth]{config/ufpr-fundo-4x3.jpg}};
}

%-----------------------------------------------------------------------
% Definições de esquema de cores.

% Ubuntu.
% http://www.color-hex.com/color-palette/2018
\definecolor{mycolor1}{HTML}{5E2750} % Título.
\definecolor{mycolor2}{HTML}{333333} % Texto.
\definecolor{mycolor3}{HTML}{DD4814} % Estrutura.
\definecolor{mycolor4}{HTML}{DD4814} % Links.
\definecolor{mycolor5}{HTML}{AEA79F} % Preenchimentos.

\hypersetup{
  colorlinks=true,
  linkcolor=mycolor4,
  urlcolor=mycolor1,
  citecolor=mycolor1
}

%-----------------------------------------------------------------------
% ATTENTION: http://www.cpt.univ-mrs.fr/~masson/latex/Beamer-appearance-cheat-sheet.pdf

\usetheme{Boadilla}
\usecolortheme{default}

\setbeamertemplate{caption}[numbered]
\setbeamertemplate{caption label separator}{: }
\setbeamertemplate{caption}[numbered]{}
\setbeamertemplate{section in toc}[sections numbered]
\setbeamertemplate{subsection in toc}[subsections numbered]
\setbeamertemplate{sections/subsections in toc}[ball]{}
\setbeamertemplate{blocks}[rounded]
\setbeamertemplate{navigation symbols}{}
\setbeamertemplate{frametitle continuation}{\gdef\beamer@frametitle{}\vspace*{1ex}}
% \setbeamertemplate{frametitle}[default][center]
% \setbeamertemplate{footline}[frame number]

\setbeamertemplate{enumerate items}[default]
\setbeamertemplate{itemize items}{\scriptsize\raise1.25pt\hbox{\donotcoloroutermaths$\blacktriangleright$}}

% Rodapé.
\setbeamercolor{title in head/foot}{parent=subsection in head/foot}
\setbeamercolor{author in head/foot}{bg=mycolor4, fg=white}
\setbeamercolor{date in head/foot}{parent=subsection in head/foot, fg=mycolor3}

% Cabeçalho.
\setbeamercolor{section in head/foot}{bg=mycolor2, fg=mycolor4}
\setbeamercolor{subsection in head/foot}{bg=mycolor2, fg=white}

\setbeamercolor{title}{fg=mycolor1}       % Título dos slides.
\setbeamercolor{titlelike}{fg=title}
\setbeamercolor{subtitle}{fg=mycolor2}    % Subtítulo.
\setbeamercolor{institute}{fg=mycolor3}   % Instituição.
\setbeamercolor{frametitle}{fg=mycolor1}  % De quadro.
\setbeamercolor{structure}{fg=mycolor3}   % Listas e rodapé.
\setbeamercolor{item projected}{bg=mycolor2}
\setbeamercolor{block title}{bg=mycolor5, fg=mycolor2}
\setbeamercolor{normal text}{fg=mycolor2} % Texto.
\setbeamercolor{caption name}{fg=normal text.fg}

% To remove empty brackets of \institution.
\makeatletter
\setbeamertemplate{footline}{
  \leavevmode%
  \hbox{%
    \begin{beamercolorbox}[
      wd=0.3\paperwidth, ht=2.25ex, dp=1ex, right]{author in head/foot}%
      \usebeamerfont{author in head/foot}\insertshortauthor{}\hspace*{1ex}
    \end{beamercolorbox}%
    \begin{beamercolorbox}[
      wd=0.6\paperwidth, ht=2.25ex, dp=1ex, left]{title in head/foot}%
      \usebeamerfont{title in head/foot}\hspace*{1ex}\insertshorttitle{}
      % \usebeamerfont{title in head/foot}\hspace*{1ex}\insertframetitle{}
    \end{beamercolorbox}%
    \begin{beamercolorbox}[
      wd=0.1\paperwidth, ht=2.25ex, dp=1ex, right]{date in head/foot}%
      \insertframenumber{}\hspace*{2ex}
    \end{beamercolorbox}
  }%
  \vskip0pt%
}
\makeatother

%-----------------------------------------------------------------------
% Knitr.

% R output e todo verbatim.
\makeatletter
\def\verbatim@font{\linespread{0.9}\normalfont\ttfamily\footnotesize}
% \def\verbatim@font{\linespread{0.3}\normalfont\ttfamily\tiny}
\makeatother

%-----------------------------------------------------------------------

\newcommand{\mytwocolumns}[4]{
  % #1: Line width fraction for the left column , e.g. 0.5.
  % #2: Line width fraction for the right column.
  % #3: Content for the left column.
  % #4: Content for the right column.
  \begin{columns}[t]
    \begin{column}{#1\linewidth} %----------- left.
      #3
    \end{column} %--------------------------- left.
    \begin{column}{#2\linewidth} %----------- right.
      #4
    \end{column} %--------------------------- right.
  \end{columns}
}

\newcommand{\myquote}[3]{
  \begin{center}
    \begin{minipage}[c]{0.19\linewidth}
      \begin{center}
        \includegraphics[height=2.7cm]{#1}
      \end{center}
    \end{minipage}
    \begin{minipage}[c]{0.7\linewidth}
      \begin{flushright}
        \textit{#2}
        \vspace{1ex}

        -- #3
      \end{flushright}
    \end{minipage}
  \end{center}
}

\newcommand{\hi}[1]{%
  \textcolor{mycolor4}{#1}\xspace
}

\newcommand{\myurl}[1]{%
  \tiny{\url{#1}}\xspace
}

%-----------------------------------------------------------------------

\author[Walmes Zeviani $\cdot$ DEST/UFPR]{
  \href{http://leg.ufpr.br/~walmes}{Prof.~Walmes Zeviani}\\
  {\small \url{walmes@ufpr.br}}
}

\institute[DEST/UFPR]{
  {Laboratório de Estatística e Geoinformação}\\
  {Departamento de Estatística}\\
  {Universidade Federal do Paraná}}

% \logo{\includegraphics[height=5em]{/home/walmes/Projects/templates/COMMON/leg.png}}
% \logo{\includegraphics[height=5em]{/home/walmes/Projects/templates/COMMON/ufpr.jpg}}

%-----------------------------------------------------------------------
